\section{Analysis of Invalid Trials}
\label{invalid}



\subsection{Results}

Invalid trials were previously defined as those trials in which the
subject pressed the space bar to end the trial without first bringing
the virtual finger to a stop. The number of invalid trials for each
subject is presented by feedback condition in Figure~12. Due to the
irregular distribution of the data, no significance test was run.
However, the figure shows two notable features. First, Subject 6 had
more invalid trials than any other subject. Second, more invalid
trials occurred under the proprioceptive-only (NV$+$P) feedback
condition than any other.



\subsection{Discussion}

Although the number of invalid trials is not directly related to task
performance, we now consider any trends that may be seen in this
information. No statistical tests were done with this data, but some
inferences can be drawn from the invalid trial counts in Figure 12.
The only obvious trend is that the NV$+$P condition appears to have
the most invalid trials, which is the case for all but two subjects.
In the post-experiment survey, one subject commented on this trend,
saying that with only proprioceptive motion feedback it was hard to
tell if the finger was moving or not. This might be a result of a
larger threshold for absolute motion detection for proprioceptive
feedback than for visual feedback. This difficulty in stopping the
finger did not appear to affect the ease of use ratings provided by
subjects, as no correlation was observed with invalid trial counts.

It is interesting to note that the no-feedback condition (NV$+$NP)
had fewer invalid trials than the proprioceptive-only condition
(NV$+$P), especially in light of the findings of Ghez et al. [1990]
that deafferented individuals tend to display endpoint drift in
non-sighted targeted reaching movements (equivalent to NV$+$NP
condition) while neurologically normal individuals do not (equivalent
to NV$+$P condition). A notable difference between our study and the
study by Ghez et al.\ is the availability of kinesthetic feedback
from the thumb pressing on the force sensor, which indicates the
magnitude of the applied force, that is, the movement command in our
study. Thus, under the no-feedback condition, subjects could use this
information to learn to apply grasping forces within the dead zone to
stop finger movement. When motion feedback is available, subjects are
likely focusing more on the feedback than on the forces applied,
since the feedback allows them to achieve better accuracy. Thus, at
the end of a trial, subjects are most likely using this feedback as
an indicator of zero velocity rather than attending to the applied
force. When visual feedback is available, it is easy to determine
whether the finger is moving or not; however, when only
proprioceptive feedback is available, the finger can be moving slowly
without the subject being aware of its motion. This explanation would
result in a larger number of failed trials for the NV$+$P condition
than for any other, as observed.